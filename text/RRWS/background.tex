% !tex root=./main.tex

\section{BACKGROUND}
\vspace*{-1ex}
\label{sec:background}

Consider data $(x^{(n)})_{n = 1}^N$ sampled from a true (unknown) generative model $p(x)$, a family of generative models $p_\theta(z, x)$ of latent variable $z$ and observation $x$ parameterized by $\theta$ and a family of inference networks $q_\phi(z \given x)$ parameterized by $\phi$.
We aim to learn the generative model by maximizing the marginal likelihood over data: \(\theta^* = \argmax_\theta  \frac{1}{N} \sum_{n = 1}^N  \log p_\theta(x^{(n)})\).
Simultaneously, we would like to learn an inference network $q_\phi(z \given x)$ that amortizes inference given observation $x$; i.e., $q_\phi(z \given x)$ maps an observation $x$ to an approximation of $p_{\theta^*}(z \given x)$.
Amortization ensures this function evaluation is cheaper than performing approximate inference of $p_{\theta^*}(z \given x)$ from scratch.
Our focus here is on such joint learning of generative model and inference network, here referred to as ``learning a deep generative model'', although we note that other approaches exist that learn the generative model~\citep{goodfellow2014generative,mohamed2016learning} or inference network~\citep{paige2016inference,le2017inference} in isolation.

We begin by reviewing \glspl{IWAE}~\citep{burda2016importance} as a general approach for learning deep generative models using \gls{SGD} methods, focusing on generative-model families with discrete latent variables, for which the na\"ive gradient estimator's high variance impedes learning.
We also review control-variate and continuous-relaxation methods for gradient-variance reduction.
\Glspl{IWAE} coupled with such gradient-variance reduction methods are currently the dominant approach for learning deep generative models with discrete latent variables.

\vspace*{-1ex}
\subsection{IMPORTANCE WEIGHTED AUTOENCODERS}
\vspace*{-1ex}

\citet{burda2016importance} introduce the \gls{IWAE}, maximizing the mean \glspl{ELBO} over data, $\frac{1}{N} \sum_{n = 1}^N \ELBO_{\text{IS}}^K(\theta, \phi, x^{(n)})$, where, for $K$ particles,
\begin{align}
  \label{eq:elbo_is}
  \ELBO_{\text{IS}}^K(\theta, \phi, x)
  &= \E_{Q_\phi(z_{1:K} \given x)}
    \!\!\left[ \log\!\left(\!\frac{1}{K} \sum_{k = 1}^K w_k \!\right)\right]\!,\\
  Q_\phi(z_{1:K} \given x)
  &= \prod_{k = 1}^K q_\phi(z_k \given x), \,w_k = \frac{p_\theta(z_k, x)}{q_\phi(z_k \given x)}.
  \nonumber
\end{align}
When $K = 1$, this reduces to the \gls{VAE}~\citep{kingma2014auto,rezende2014stochastic}.
\citet{burda2016importance} show that $\ELBO_{\text{IS}}^K(\theta, \phi, x)$ is a lower bound on $\log p_\theta(x)$ and that increasing~$K$ leads to a tighter lower bound.
%
Further, tighter lower bounds arising from increasing~$K$ improve learning of the generative model, but impair learning of the inference network~\citep{rainforth2018tighter}, as the signal-to-noise ratio of~\(\theta\)'s gradient estimator is $O(\sqrt{K})$ whereas~\(\phi\)'s is $O(1 / \sqrt{K})$.
%
Note that although \citet{tucker2019doubly} solve this for reparameterizable distributions, the issue persists for discrete distributions.
%
Consequently, poor learning of the inference network, beyond a certain point (large~\(K\)), can actually impair learning of the generative model as well; a finding we explore in \cref{sec:experiments/gmm}.

Optimizing the \gls{IWAE} objective using \gls{SGD} methods requires unbiased gradient estimators of $\ELBO_{\text{IS}}^K(\theta, \phi, x)$ with respect to $\theta$ and $\phi$~\citep{robbins1951stochastic}.
%
$\nabla_\theta \ELBO_{\text{IS}}^K(\theta, \phi, x)$ is estimated by evaluating $\nabla_\theta \log \hat Z_K$ using samples $z_{1:K} \sim Q_\phi(\cdot \given x)$, where $\hat Z_K = \frac{1}{K} \sum_{k = 1}^K\! w_k$.
$\nabla_\phi \!\ELBO_{\acrshort{IS}}^K\!(\theta, \phi, x)$ is estimated similarly for models with reparameterizable latents, discrete (and other non-reparameterizable) latents require the \acrshort{REINFORCE} gradient estimator~\citep{williams1992simple}
% which, given $z_{1:K} \sim Q_\phi(\cdot \given x)$, is:
% For models with continuous latent variables $\nabla_\phi \ELBO_{\text{IS}}^K(\theta, \phi, x)$ is estimated similarly, however one must resort to the the \acrshort{REINFORCE} trick~\citep{williams1992simple} estimator for models with discrete latents:
% For models with discrete latent variables, $\nabla_\phi \ELBO_{\text{IS}}^K(\theta, \phi, x)$ is estimated using the \acrshort{REINFORCE} trick~\citep{williams1992simple}
\begin{align}
  \!\!\!\!g_{\acrshort{REINFORCE}}
  \!=\! \underbrace{\log \hat Z_K \nabla_\phi \log Q_\phi(z_{1:K} \given x)}_{\circled[5]{1}}
  \!+\! \underbrace{\nabla_\phi \log \hat Z_K}_{\circled[5]{2}}.\!\!
    \label{eq:iwae-reinforce}
    % \left(\!\frac{1}{K} \sum_{k = 1}^K w_k\!\right) \left(\frac{1}{K} \sum_{k = 1}^K w_k\right)
\end{align}

\vspace*{-1ex}
\subsection{CONTINUOUS RELAXATIONS AND CONTROL VARIATES}
\vspace*{-1ex}
\label{sec:background/control-variates}

Since the gradient estimator in \cref{eq:iwae-reinforce} typically suffers from high variance, mainly due to the effect of \circled[5]{1}, a number of approaches have been developed to ameliorate the issue.
%
These can be broadly categorized into approaches that directly transform the discrete latent variables (continuous relaxations), or approaches that target improvement of the na\"ive \acrshort{REINFORCE} estimator (control variates).

\paragraph{Continuous Relaxations:}%
%
Here, discrete variables are transformed to enable reparameterization~\citep{kingma2014auto,rezende2014stochastic}, helping reduce gradient-estimator variance.
% sid: does the variance-reduction bit have a citation?
%
Approaches span the Gumbel distribution~\citep{maddison2017concrete,jang2017categorical}, spike-and-X transforms~\citep{rolfe2016dvae}, overlapping exponentials~\citep{vahdat2018dvaepp}, and generalized overlapping exponentials for tighter bounds~\citep{vahdat2018dvaehash}.

Besides difficulties inherent to such methods, such as tuning temperature parameters, or the suitability of undirected Boltzmann machine priors, these methods are not well suited for learning \glspl{SCFM} as they generate samples on the surface of a probability simplex rather than its vertices.
%
For example, sampling from a transformed Bernoulli distribution yields samples of the form \([\alpha, (1 - \alpha)]\) rather than simply 0 or 1---the latter form required for branching.
%
With relaxed samples, as illustrated in \cref{fig:exponential-all}, one would need to execute \emph{all} the exponentially many discrete-variable driven branches in the model, weighting each branch appropriately---something that can quickly become infeasible for even moderately complex models.
%
However, for purposes of comparison, for relatively simple \glspl{SCFM}, one could apply methods involving continuous relaxations, as demonstrated in \cref{sec:experiments/gmm}.

\paragraph{Control Variates:}%
%
Here, approaches build on the \acrshort{REINFORCE} estimator for the \gls{IWAE} \gls{ELBO} objective, designing control-variate schemes to reduce the variance of the na\"ive estimator.
%
\Gls{VIMCO}~\citep{mnih2016variational} eschews designing an explicit control variate, instead exploiting the particle set obtained in \gls{IWAE}.
%
It replaces \circled[5]{1} with
  % g_{\acrshort{VIMCO}}^{\circled[2]{1}}
  % &\!=\! \sum_{\ell = 1}^K \biggl(
  % \underbrace{\log \hat Z_K}_{A}
  % - \underbrace{\log \frac{1}{K} \biggl(e^{\frac{1}{K - 1} \!\!\sum\limits_{k \neq \ell} \!\log w_k}
  % \!\!+ \sum_{k \neq \ell} w_k \biggr)\!}_{B}
  % \biggr)\\
  % &\quad \nabla_\phi \log q_\phi(z_\ell \given x),
%
\begin{align}
\vspace*{-1ex}
  g_{\acrshort{VIMCO}}^{\circled[2]{1}}
  &= \sum_{k = 1}^K (\log \hat Z_K - \Upsilon_{-k}) \nabla_\phi \log q_\phi(z_k \given x),\\[-0.5ex]
  \Upsilon_{-k}
  & = \log \frac{1}{K} \biggl(\exp\biggl(\frac{1}{K - 1} \sum\limits_{\ell \neq k} \log w_\ell\biggr)
    + \sum_{\ell \neq k} w_\ell \biggr) \nonumber
\vspace*{-1ex}
\end{align}
%
where \(\Upsilon_{-k} \perp\hspace*{-6pt}\perp z_k\) and highly correlated with $\log \hat Z_K$.

Finally, assuming $z_k$ is a discrete random variable with $C$ categories\footnote{The assumption is needed only for notational convenience. However, using more structured latents leads to difficulties in picking the control-variate architecture.}, \acrshort{REBAR}~\citep{tucker2017rebar} and \acrshort{RELAX}~\citep{grathwohl2018backpropagation} improve on \citet{mnih2014neural} and \citet{gu2016muprop}, replacing \circled[5]{1} as
%
\begin{align}
  g_{\acrshort{RELAX}}^{\circled[2]{1}}
  &= \biggl(\log \hat Z_K - c_\rho(\tilde g_{1:K}) \biggr) \nabla_\phi \log Q_\phi(z_{1:K} \given x) \nonumber \\
  &\quad+ \nabla_\phi c_\rho(g_{1:K}) - \nabla_\phi c_\rho(\tilde g_{1:K}),
\end{align}
%
where $g_k$ is a $C$-dimensional vector of reparameterized Gumbel random variates, $z_k$ is a one-hot argmax function of $g_k$, and $\tilde g_k$ is a vector of reparameterized conditional Gumbel random variates conditioned on $z_k$.
The conditional Gumbel random variates are a form of Rao-Blackwellization used to reduce variance.
The control variate $c_\rho$, parameterized by $\rho$, is optimized to minimize the gradient variance estimates along with the main \gls{ELBO} optimization, leading to state-of-the-art performance on, for example, sigmoid belief networks~\citep{neal1992connectionist}.
The main difficulty in using this method is choosing a suitable family of $c_\rho$, as some choices lead to higher variance despite concurrent gradient-variance minimization.

% Moreover, the objective for the concurrent optimization requires evaluating a Jacobian-vector product that induces an overhead of $O(D_\phi D_\rho)$~\tal{check this} where $D_\phi, D_\rho$ are number of inference network and control-variate parameters respectively.


%%% Local Variables:
%%% mode: latex
%%% TeX-master: "main"
%%% End:

% LocalWords:  parameterized IWAE SGD ive Autoencoders ELBO VAE latents
% LocalWords:  reparameterizable reparameterisation SCFM VIMCO reparameterized
% LocalWords:  Gumbel argmax Blackwellization
