% !Tex root=tb_icml_2018.tex

\section{New Estimators}
\label{sec:algs}

Based on our theoretical results, we now introduce three new
algorithms that address the issue of diminishing \gls{SNR} for the inference
network.  Our first, \gls{MIWAE}, is exactly equivalent to the 
general formulation given in~\eqref{eq:grad-est}, the distinction from previous
approaches coming from the fact that it takes both $M>1$ and $K>1$.  
The motivation for this is that because our inference network \gls{SNR} increases as
$O(\sqrt{M/K})$, we should be able to mitigate the issues increasing
$K$ has on the \gls{SNR} by also increasing $M$.  For fairness, we will
keep our overall budget $T=MK$ fixed, but we will show that given this
budget, the optimal value for $M$ is often not $1$.  In practice, we expect that
it will often be beneficial to increase the mini-batch size $N$ rather than $M$ for
\gls{MIWAE};
as we showed in Section~\ref{sec:multi} this has the same effect on the~\gls{SNR}.  Nonetheless, \gls{MIWAE} forms an interesting reference method
for testing our theoretical results and, as we will show, it can offer improvements
over \gls{IWAE} for a given $N$.

Our second algorithm, \gls{CIWAE} uses a convex combination of the
\gls{IWAE} and \gls{VAE} bounds, namely
\begin{align}
\ELBO_{\text{CIWAE}} = \beta \ELBO_{\text{VAE}} + (1-\beta) \ELBO_{\text{IWAE}}
\end{align}
where $\beta \in [0,1]$ is a combination parameter.  It is trivial
to see that $\ELBO_{\text{CIWAE}}$ is a lower bound on the log marginal
that is tighter than the \gls{VAE} bound but looser than the \gls{IWAE}
bound.  We then employ the following estimator
\begin{align}
\label{eq:CIWAE-est}
\Delta_{K,\beta}^{\text{C}}
=& \\\nabla_{\theta, \phi} &\left(
\beta \frac{1}{K} \sum_{k=1}^{K} 
\log w_k + (1-\beta) \log \left(\frac{1}{K} \sum_{k=1}^{K}  w_k\right)\right) \nonumber
\end{align}
where we use the same $w_k$ for both terms.  The motivation
for  \gls{CIWAE} is that, if we set $\beta$ to a relatively small value, the
objective will behave mostly like \gls{IWAE}, except when the
expected \gls{IWAE} gradient becomes very small.  When this happens,
the \gls{VAE} component should ``take-over'' and alleviate SNR issues:
the asymptotic SNR of $\Delta_{K,\beta}^{\text{C}}$ for $\phi$ is
$O(\sqrt{MK})$ because the \gls{VAE} component has non-zero expectation
in the limit $K\rightarrow\infty$.

Our results suggest that what is good for the generative
network, in terms of setting $K$, is often detrimental for the inference 
network.  It is therefore natural to question whether it is sensible
to always use the same target for both the inference and generative networks.
Motivated by this,
our third method, \gls{PIWAE}, uses the \gls{IWAE} target when
training the generative network, but the \gls{MIWAE} target for training
the inference network.  We thus have
\begin{subequations}
\begin{align}
\label{eq:PIWAE-est}
\Delta_{M,K}^{\text{P}} (\theta) &= \nabla_{\theta} \log \frac{1}{K}
\sum\nolimits_{k=1}^{K} w_k \\
\Delta_{M,K}^{\text{P}} (\phi) &= \frac{1}{M} \sum\nolimits_{m=1}^{M}
\nabla_{\phi} \log \frac{1}{L} \sum\nolimits_{\ell=1}^{L} w_{m,\ell}
\end{align}
\end{subequations}
where we will generally set $K=ML$ so that the same weights can be
used for both gradients.  