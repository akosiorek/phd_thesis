% !Tex root=./main.tex

\vspace{-4pt}

\section{Derivation of Optimal Parameters for Gaussian Experiment}
\label{sec:optGauss}

\vspace{-4pt}

To derive the optimal parameters for the Gaussian experiment we first note that
\begin{align*}
\mathcal J(\theta, \phi) %&:= \frac{1}{N} \sum_{n = 1}^N \ELBO_{\text{IS}}(\theta, \phi, x^{(n)}) \\
&= \frac{1}{N}\log \prod_{n=1}^{N} p_{\theta}(x^{(n)}) - \frac{1}{N}\sum_{n=1}^{N} \KL{Q_{\phi}(z_{1:K}|x^{(n)})}{P_{\theta}(z_{1:K}|x^{(n)})} \quad \text{where}\\
P_{\theta}(z_{1:K}|x^{(n)}) &= \frac{1}{K} \sum_{k=1}^{K}
q_{\phi} (z_1 | x^{(n)}) \dots q_{\phi} (z_{k-1} | x^{(n)}) p_{\theta} (z_k | x^{(n)}) 
q_{\phi} (z_{k+1} | x^{(n)}) \dots q_{\phi} (z_{K} | x^{(n)}),
\end{align*}
$Q_{\phi}(z_{1:K}|x^{(n)})$ is as per~\eqref{eq:background/q_is_z_is} 
 and the form of the \gls{KL} is taken from~\cite{le2017auto}.
Next, we note that $\phi$ only controls the mean of the proposal so, while it is not possible to drive the
$\textsc{KL}$ to zero, it will be minimized for any particular $\theta$ when the means of $q_{\phi}(z|x^{(n)})$
and $p_{\theta}(z|x^{(n)})$ are the same.  
Furthermore, the corresponding minimum possible value of the \textsc{KL} is independent of
$\theta$ and so we can
calculate the optimum pair $(\theta^*,\phi^*)$ by first optimizing for $\theta$ and then choosing the matching $\phi$.
The optimal $\theta$ maximizes $\log \prod_{n=1}^{N} p_{\theta}(x^{(n)})$, giving $\theta^* := \mu^* = \frac{1}{N} \sum_{n = 1}^N x^{(n)}$.
As we straightforwardly have $p_{\theta} (z | x^{(n)}) = 
\mathcal{N}(z; \left(x^{(n)}+\mu\right)/2, I/2)$, the \text{KL} is then minimized
when $A=I/2$ and $b=\mu/2$, giving $\phi^* := (A^*, b^*)$, where $A^* = I / 2$ and $b^* = \mu^* / 2$.

\vspace{-4pt}