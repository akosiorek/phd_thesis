\newcommand{\ak}[1]{\textcolor{blue}{\textbf{ak}: #1}}
\newcommand{\yw}[1]{\textcolor{pink}{\textbf{yw}: #1}}
\newcommand{\hjk}[1]{\textcolor{green}{\textbf{hjk}: #1}}
\newcommand{\hip}[1]{\textcolor{orange}{\textbf{hip}: #1}}

\newcommand{\AK}[1]{\todo{\ak{#1}}}

\newcommand*{\B}[1]{\ifmmode\bm{#1}\else\textbf{#1}\fi}

\newcommand{\MLP}{\operatorname{MLP}}
\newcommand{\RNN}{\operatorname{RNN}}
\newcommand{\STN}{\operatorname{ST}}
\newcommand{\bern}{\operatorname{Bernoulli}}
\newcommand{\cat}{\operatorname{Categorical}}

\DeclarePairedDelimiter{\fences}{(}{)}
\DeclarePairedDelimiter{\norm}{\lVert}{\rVert}

\newcommand{\LSTM}[1]{ \mathrm{LSTM} \fences{#1} }
\newcommand{\flatten}[1]{ \mathrm{vec} \fences{#1} }
\newcommand{\reg}[1]{ \ensuremath{R} \fences{#1} }


\definecolor{pink}{HTML}{ff00ff}
\definecolor{orange}{HTML}{ff9900}
\definecolor{blue}{HTML}{0000ff}
\definecolor{darkgreen}{rgb}{0,.502,0}


\SetKw{Continue}{continue}


\newcommand{\hT}[2]{\textcolor{blue}{\bm{h}_{#1}^{T, #2}}}
\newcommand{\hR}[2]{\textcolor{orange}{\bm{h}_{#1}^{R, #2}}}
\newcommand{\hD}[2]{\textcolor{pink}{\bm{h}_{#1}^{D, #2}}}

\newcommand{\RT}{\textcolor{blue}{\operatorname{R}_\phi^T}}
\newcommand{\Rr}{\textcolor{orange}{\operatorname{R}_\phi^R}}
\newcommand{\RD}{\textcolor{pink}{\operatorname{R}_\phi^D}}


\newcommand{\sidecaption}[1]%
{\raisebox{\abovecaptionskip}{\begin{subfigure}[t]{1.6em}
		\caption[singlelinecheck=off]{}%
		\label{#1}
\end{subfigure}}\ignorespaces}


\AtBeginEnvironment{subappendices}{%
	\chapter*{Appendix}
	\addcontentsline{toc}{chapter}{Appendices}
	\counterwithin{figure}{section}
	\counterwithin{table}{section}
}

\probdists{p,q}
\probdists[pd]{p^D}
\probdists[pp]{p^P}
\probdists[qd]{q^D}
\probdists[qp]{q^P}

\variables{a,b,c,e,g,l,t,o,s,v,w,x,y,z,D,P,S}
\variables{A,I}

\variables[app]{\alpha}
\variables[mean]{\mu}
\variables[std]{\sigma}
\variables[map]{\nu}
\variables[dparam]{\psi}


\newcommand{\what}{\textsc{what} }
\newcommand{\where}{\textsc{where} }
\newcommand{\pres}{\textsc{presence} }

\newcommand{\E}{\mathbb{E}}
\newcommand{\KL}{D_{\mathrm{KL}}}
\newcommand{\given}{\lvert}
\DeclareMathOperator*{\argmax}{arg\,max}
\DeclareMathOperator*{\argmin}{arg\,min}
\DeclareMathOperator{\ELBO}{\acrshort{ELBO}}
\DeclareMathOperator{\std}{\mathrm{std}}
\DeclareMathOperator{\SNR}{\acrshort{SNR}}

\newcommand{\circled}[2][]{%
  \tikz[baseline=(char.base)]{%
    \node[shape = circle, draw, inner sep = 1pt,scale=0.75]
    (char) {\phantom{\ifblank{#1}{#2}{#1}}};%
    \node at (char.center) {\makebox[0pt][c]{\scriptsize #2}};}}
\robustify{\circled}

\theoremstyle{plain}
\newtheorem{theorem}{Theorem}

\theoremstyle{plain}
\newtheorem{theoremApp}{Theorem}

\theoremstyle{remark}
\newtheorem{remark}{Remark}


\setlength{\parindent}{0pt}