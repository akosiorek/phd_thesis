\section{Architecture Details}
\label{sec:mohart_architecture_details}
The architecture details were chosen to optimise \textsc{hart} performance on the MOTChallenge dataset. They deviate from the original \textsc{hart} implementation (\Cref{ch:hart}) as follows: A presence variable predicts whether an object is in the scene and successfully tracked. This is trained with a binary cross entropy loss. The maximum number of objects to be tracked simultaneously was set to 5 for the UA-DETRAC and MOTChallenge dataset. For the more crowded Stanford drone dataset, this number was set to 10. The feature extractor is a three layer convolutional network with a kernel size of 5, a stride of 2 in the first and last layer, 32 channels in the first two layers, 64 channels in the last layer, ELU activations, and skip connections. This converts the initial $32 \times 32 \times 3$ glimpse into a $7 \times 7 \times 64$ feature representation. This is followed by a fully connected layer with a 128 dimensional output and an elu activation. The spatial attention parameters are linearly projected onto 128 dimensions and added to this feature representation serving as a positional encoding. The LSTM has a hidden state size of 128. The self-attention unit in \textsc{mohart} comprises linear projects the inputs to dimensionality 128 for each keys, queries and values. For the real-world experiments, in addition to the extracted features from the glimpse, the hidden states from the previous LSTM state are also fed as an input by concatinating them with the features. In all cases, the output of the attention module is concatenated to the input features of the respective object.

As an optimizer, we used RMSProp with momentum set to $0.9$ and learning rate $5*10^{-6}$. For the MOTChallenge dataset and the UA-DETRAC dataset, the models were trained for 100,000 iterations of batch size 10 and the reported IoU is exponentially smoothed over iterations to achieve lower variance. For the Stanford Drone dataset, the batch size was increased to 32, reducing time to convergence and hence model training to 50,000 iterations.
