%Your abstract text goes here.  Check your departmental regulations, but generally, this should be less than 300 words. 

Whenever an agent interacts with its environment, it has to take into account and interact with any objects present in this environment.
And yet, the majority of machine learning solutions either treat objects only implicitly or employ highly-engineered solutions that account for objects through object detection algorithms.
In this thesis, we explore supervised and unsupervised methods for learning object-centric representations from vision.
We focus on end-to-end learning, where information about objects can be extracted directly from images, and where every object can be separately described by a single vector-valued variable.
Specifically, we present three novel methods:
\begin{itemize}
	\item \textsc{Hart} and \textsc{mohart}, which track single- and multiple-objects in video, respectively, by using \textsc{rnn}s with a hierarchy of differentiable attention mechanisms.
	These algorithms learn to anticipate future appearance changes and movement of tracking objects, thereby learning representations that describe every tracked object separately.
	\item \textsc{Sqair}, a \textsc{vae}-based generative model of moving objects, which explicitly models disappearance and appearance of new objects in the scene.
	It models every object with a separate latent variable, and disentangles appearance, position and scale of each object.
	Posterior inference in this model allows for unsupervised object detection and tracking.
	\item \textsc{Scae}, an unsupervised autoencoder with in-built knowledge of two-dimensional geometry and object-part decomposition, which is based on capsule networks.
	It learns to discover parts present in an image, and group those parts into objects.
	Each object is modelled by a separate \textit{object capsule}, whose activation probability is highly correlated with the object class, therefore allowing for state-of-the-art unsupervised image classification. 
\end{itemize}